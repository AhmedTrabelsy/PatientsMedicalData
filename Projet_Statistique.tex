% Options for packages loaded elsewhere
\PassOptionsToPackage{unicode}{hyperref}
\PassOptionsToPackage{hyphens}{url}
\documentclass[
]{article}
\usepackage{xcolor}
\usepackage[margin=1in]{geometry}
\usepackage{amsmath,amssymb}
\setcounter{secnumdepth}{-\maxdimen} % remove section numbering
\usepackage{iftex}
\ifPDFTeX
  \usepackage[T1]{fontenc}
  \usepackage[utf8]{inputenc}
  \usepackage{textcomp} % provide euro and other symbols
\else % if luatex or xetex
  \usepackage{unicode-math} % this also loads fontspec
  \defaultfontfeatures{Scale=MatchLowercase}
  \defaultfontfeatures[\rmfamily]{Ligatures=TeX,Scale=1}
\fi
\usepackage{lmodern}
\ifPDFTeX\else
  % xetex/luatex font selection
\fi
% Use upquote if available, for straight quotes in verbatim environments
\IfFileExists{upquote.sty}{\usepackage{upquote}}{}
\IfFileExists{microtype.sty}{% use microtype if available
  \usepackage[]{microtype}
  \UseMicrotypeSet[protrusion]{basicmath} % disable protrusion for tt fonts
}{}
\makeatletter
\@ifundefined{KOMAClassName}{% if non-KOMA class
  \IfFileExists{parskip.sty}{%
    \usepackage{parskip}
  }{% else
    \setlength{\parindent}{0pt}
    \setlength{\parskip}{6pt plus 2pt minus 1pt}}
}{% if KOMA class
  \KOMAoptions{parskip=half}}
\makeatother
\usepackage{color}
\usepackage{fancyvrb}
\newcommand{\VerbBar}{|}
\newcommand{\VERB}{\Verb[commandchars=\\\{\}]}
\DefineVerbatimEnvironment{Highlighting}{Verbatim}{commandchars=\\\{\}}
% Add ',fontsize=\small' for more characters per line
\usepackage{framed}
\definecolor{shadecolor}{RGB}{248,248,248}
\newenvironment{Shaded}{\begin{snugshade}}{\end{snugshade}}
\newcommand{\AlertTok}[1]{\textcolor[rgb]{0.94,0.16,0.16}{#1}}
\newcommand{\AnnotationTok}[1]{\textcolor[rgb]{0.56,0.35,0.01}{\textbf{\textit{#1}}}}
\newcommand{\AttributeTok}[1]{\textcolor[rgb]{0.13,0.29,0.53}{#1}}
\newcommand{\BaseNTok}[1]{\textcolor[rgb]{0.00,0.00,0.81}{#1}}
\newcommand{\BuiltInTok}[1]{#1}
\newcommand{\CharTok}[1]{\textcolor[rgb]{0.31,0.60,0.02}{#1}}
\newcommand{\CommentTok}[1]{\textcolor[rgb]{0.56,0.35,0.01}{\textit{#1}}}
\newcommand{\CommentVarTok}[1]{\textcolor[rgb]{0.56,0.35,0.01}{\textbf{\textit{#1}}}}
\newcommand{\ConstantTok}[1]{\textcolor[rgb]{0.56,0.35,0.01}{#1}}
\newcommand{\ControlFlowTok}[1]{\textcolor[rgb]{0.13,0.29,0.53}{\textbf{#1}}}
\newcommand{\DataTypeTok}[1]{\textcolor[rgb]{0.13,0.29,0.53}{#1}}
\newcommand{\DecValTok}[1]{\textcolor[rgb]{0.00,0.00,0.81}{#1}}
\newcommand{\DocumentationTok}[1]{\textcolor[rgb]{0.56,0.35,0.01}{\textbf{\textit{#1}}}}
\newcommand{\ErrorTok}[1]{\textcolor[rgb]{0.64,0.00,0.00}{\textbf{#1}}}
\newcommand{\ExtensionTok}[1]{#1}
\newcommand{\FloatTok}[1]{\textcolor[rgb]{0.00,0.00,0.81}{#1}}
\newcommand{\FunctionTok}[1]{\textcolor[rgb]{0.13,0.29,0.53}{\textbf{#1}}}
\newcommand{\ImportTok}[1]{#1}
\newcommand{\InformationTok}[1]{\textcolor[rgb]{0.56,0.35,0.01}{\textbf{\textit{#1}}}}
\newcommand{\KeywordTok}[1]{\textcolor[rgb]{0.13,0.29,0.53}{\textbf{#1}}}
\newcommand{\NormalTok}[1]{#1}
\newcommand{\OperatorTok}[1]{\textcolor[rgb]{0.81,0.36,0.00}{\textbf{#1}}}
\newcommand{\OtherTok}[1]{\textcolor[rgb]{0.56,0.35,0.01}{#1}}
\newcommand{\PreprocessorTok}[1]{\textcolor[rgb]{0.56,0.35,0.01}{\textit{#1}}}
\newcommand{\RegionMarkerTok}[1]{#1}
\newcommand{\SpecialCharTok}[1]{\textcolor[rgb]{0.81,0.36,0.00}{\textbf{#1}}}
\newcommand{\SpecialStringTok}[1]{\textcolor[rgb]{0.31,0.60,0.02}{#1}}
\newcommand{\StringTok}[1]{\textcolor[rgb]{0.31,0.60,0.02}{#1}}
\newcommand{\VariableTok}[1]{\textcolor[rgb]{0.00,0.00,0.00}{#1}}
\newcommand{\VerbatimStringTok}[1]{\textcolor[rgb]{0.31,0.60,0.02}{#1}}
\newcommand{\WarningTok}[1]{\textcolor[rgb]{0.56,0.35,0.01}{\textbf{\textit{#1}}}}
\usepackage{longtable,booktabs,array}
\usepackage{calc} % for calculating minipage widths
% Correct order of tables after \paragraph or \subparagraph
\usepackage{etoolbox}
\makeatletter
\patchcmd\longtable{\par}{\if@noskipsec\mbox{}\fi\par}{}{}
\makeatother
% Allow footnotes in longtable head/foot
\IfFileExists{footnotehyper.sty}{\usepackage{footnotehyper}}{\usepackage{footnote}}
\makesavenoteenv{longtable}
\usepackage{graphicx}
\makeatletter
\newsavebox\pandoc@box
\newcommand*\pandocbounded[1]{% scales image to fit in text height/width
  \sbox\pandoc@box{#1}%
  \Gscale@div\@tempa{\textheight}{\dimexpr\ht\pandoc@box+\dp\pandoc@box\relax}%
  \Gscale@div\@tempb{\linewidth}{\wd\pandoc@box}%
  \ifdim\@tempb\p@<\@tempa\p@\let\@tempa\@tempb\fi% select the smaller of both
  \ifdim\@tempa\p@<\p@\scalebox{\@tempa}{\usebox\pandoc@box}%
  \else\usebox{\pandoc@box}%
  \fi%
}
% Set default figure placement to htbp
\def\fps@figure{htbp}
\makeatother
\setlength{\emergencystretch}{3em} % prevent overfull lines
\providecommand{\tightlist}{%
  \setlength{\itemsep}{0pt}\setlength{\parskip}{0pt}}
\usepackage{bookmark}
\IfFileExists{xurl.sty}{\usepackage{xurl}}{} % add URL line breaks if available
\urlstyle{same}
\hypersetup{
  pdftitle={Projet Statistique : Analyse des Données Médicales},
  pdfauthor={Ton Nom},
  hidelinks,
  pdfcreator={LaTeX via pandoc}}

\title{Projet Statistique : Analyse des Données Médicales}
\author{Ton Nom}
\date{13/12/2025}

\begin{document}
\maketitle

{
\setcounter{tocdepth}{2}
\tableofcontents
}
\section{1. Introduction}\label{introduction}

\subsection{1.1 Contexte}\label{contexte}

Ce projet vise à étudier un jeu de données médicales contenant des
informations sur les patients : âge, sexe, poids, tension artérielle,
cholestérol, groupe de traitement, durée de suivi et score de symptômes.

\subsection{1.2 Problématique}\label{probluxe9matique}

\begin{quote}
Les caractéristiques cliniques (âge, poids, tension, cholestérol) et le
traitement influencent-ils significativement l'état des patients (score
de symptômes) ?
\end{quote}

\subsection{1.3 Objectifs}\label{objectifs}

\begin{itemize}
\tightlist
\item
  Décrire les variables (analyse univariée)
\item
  Étudier les relations entre variables (analyse bivariée)
\item
  Construire un modèle explicatif (analyse multivariée)
\item
  Tester les hypothèses statistiques
\end{itemize}

\section{2. Préparation des
données}\label{pruxe9paration-des-donnuxe9es}

\subsection{2.1 Chargement des packages}\label{chargement-des-packages}

\begin{Shaded}
\begin{Highlighting}[]
\CommentTok{\# Installer les packages si nécessaires :}
\CommentTok{\# install.packages(c("tidyverse", "skimr", "GGally", "psych", "broom"))}

\FunctionTok{library}\NormalTok{(tidyverse)}
\FunctionTok{library}\NormalTok{(skimr)}
\FunctionTok{library}\NormalTok{(GGally)}
\FunctionTok{library}\NormalTok{(psych)}
\FunctionTok{library}\NormalTok{(broom)}
\end{Highlighting}
\end{Shaded}

\subsection{2.2 Importation du jeu de
données}\label{importation-du-jeu-de-donnuxe9es}

\begin{Shaded}
\begin{Highlighting}[]
\NormalTok{data }\OtherTok{\textless{}{-}} \FunctionTok{read.csv}\NormalTok{(}\FunctionTok{file.choose}\NormalTok{(),}
                 \AttributeTok{sep =} \StringTok{";"}\NormalTok{,}
                 \AttributeTok{header =} \ConstantTok{TRUE}\NormalTok{,}
                 \AttributeTok{stringsAsFactors =} \ConstantTok{FALSE}\NormalTok{,}
                 \AttributeTok{na.strings =} \FunctionTok{c}\NormalTok{(}\StringTok{""}\NormalTok{, }\StringTok{"NA"}\NormalTok{, }\StringTok{"NaN"}\NormalTok{, }\StringTok{"NULL"}\NormalTok{, }\StringTok{"?"}\NormalTok{))}

\CommentTok{\# 👀 Aperçu rapide du jeu de données}
\FunctionTok{head}\NormalTok{(data)}
\end{Highlighting}
\end{Shaded}

\begin{verbatim}
##   ID Age Sexe Poids Tension Cholesterol Groupe_Traitement Suivi_Jours
## 1  1  63    F  48.0     130         188                 B          20
## 2  2  76    F  83.4     136         193                 B          20
## 3  3  53    M  68.4     128         200                 B          27
## 4  4  39    F  55.7     143         207                 B          21
## 5  5  67    F  63.8     150         171                 B          27
## 6  6  32    F  49.0     110          NA                 B          26
##   Symptom_Score
## 1             6
## 2             8
## 3             1
## 4            10
## 5             4
## 6             1
\end{verbatim}

\begin{Shaded}
\begin{Highlighting}[]
\FunctionTok{str}\NormalTok{(data)}
\end{Highlighting}
\end{Shaded}

\begin{verbatim}
## 'data.frame':    100 obs. of  9 variables:
##  $ ID               : int  1 2 3 4 5 6 7 8 9 10 ...
##  $ Age              : int  63 76 53 39 67 32 45 63 43 47 ...
##  $ Sexe             : chr  "F" "F" "M" "F" ...
##  $ Poids            : num  48 83.4 68.4 55.7 63.8 49 64.8 81.3 65.1 57.1 ...
##  $ Tension          : num  130 136 128 143 150 110 143 136 130 106 ...
##  $ Cholesterol      : num  188 193 200 207 171 NA 181 249 NA 242 ...
##  $ Groupe_Traitement: chr  "B" "B" "B" "B" ...
##  $ Suivi_Jours      : int  20 20 27 21 27 26 29 29 21 25 ...
##  $ Symptom_Score    : int  6 8 1 10 4 1 8 1 3 4 ...
\end{verbatim}

\subsection{2.3 Nettoyage des données}\label{nettoyage-des-donnuxe9es}

\begin{Shaded}
\begin{Highlighting}[]
\NormalTok{data }\OtherTok{\textless{}{-}}\NormalTok{ data }\SpecialCharTok{\%\textgreater{}\%}
  \FunctionTok{rename}\NormalTok{(}
    \AttributeTok{ID =}\NormalTok{ ID,}
    \AttributeTok{Age =}\NormalTok{ Age,}
    \AttributeTok{Sexe =}\NormalTok{ Sexe,}
    \AttributeTok{Poids =}\NormalTok{ Poids,}
    \AttributeTok{Tension =}\NormalTok{ Tension,}
    \AttributeTok{Cholesterol =}\NormalTok{ Cholesterol,}
    \AttributeTok{Groupe\_Traitement =}\NormalTok{ Groupe\_Traitement,}
    \AttributeTok{Suivi\_Jours =}\NormalTok{ Suivi\_Jours,}
    \AttributeTok{Symptom\_Score =}\NormalTok{ Symptom\_Score}
\NormalTok{  ) }\SpecialCharTok{\%\textgreater{}\%}
  \FunctionTok{mutate}\NormalTok{(}
    \AttributeTok{Sexe =} \FunctionTok{as.factor}\NormalTok{(Sexe),}
    \AttributeTok{Groupe\_Traitement =} \FunctionTok{as.factor}\NormalTok{(Groupe\_Traitement)}
\NormalTok{  )}

\CommentTok{\# Suppression des doublons}
\NormalTok{data }\OtherTok{\textless{}{-}} \FunctionTok{distinct}\NormalTok{(data)}

\CommentTok{\# Résumé rapide}
\FunctionTok{skim}\NormalTok{(data)}
\end{Highlighting}
\end{Shaded}

\begin{longtable}[]{@{}ll@{}}
\caption{Data summary}\tabularnewline
\toprule\noalign{}
\endfirsthead
\endhead
\bottomrule\noalign{}
\endlastfoot
Name & data \\
Number of rows & 100 \\
Number of columns & 9 \\
\_\_\_\_\_\_\_\_\_\_\_\_\_\_\_\_\_\_\_\_\_\_\_ & \\
Column type frequency: & \\
factor & 2 \\
numeric & 7 \\
\_\_\_\_\_\_\_\_\_\_\_\_\_\_\_\_\_\_\_\_\_\_\_\_ & \\
Group variables & None \\
\end{longtable}

\textbf{Variable type: factor}

\begin{longtable}[]{@{}
  >{\raggedright\arraybackslash}p{(\linewidth - 10\tabcolsep) * \real{0.2500}}
  >{\raggedleft\arraybackslash}p{(\linewidth - 10\tabcolsep) * \real{0.1389}}
  >{\raggedleft\arraybackslash}p{(\linewidth - 10\tabcolsep) * \real{0.1944}}
  >{\raggedright\arraybackslash}p{(\linewidth - 10\tabcolsep) * \real{0.1111}}
  >{\raggedleft\arraybackslash}p{(\linewidth - 10\tabcolsep) * \real{0.1250}}
  >{\raggedright\arraybackslash}p{(\linewidth - 10\tabcolsep) * \real{0.1806}}@{}}
\toprule\noalign{}
\begin{minipage}[b]{\linewidth}\raggedright
skim\_variable
\end{minipage} & \begin{minipage}[b]{\linewidth}\raggedleft
n\_missing
\end{minipage} & \begin{minipage}[b]{\linewidth}\raggedleft
complete\_rate
\end{minipage} & \begin{minipage}[b]{\linewidth}\raggedright
ordered
\end{minipage} & \begin{minipage}[b]{\linewidth}\raggedleft
n\_unique
\end{minipage} & \begin{minipage}[b]{\linewidth}\raggedright
top\_counts
\end{minipage} \\
\midrule\noalign{}
\endhead
\bottomrule\noalign{}
\endlastfoot
Sexe & 0 & 1 & FALSE & 2 & M: 62, F: 38 \\
Groupe\_Traitement & 0 & 1 & FALSE & 2 & A: 58, B: 42 \\
\end{longtable}

\textbf{Variable type: numeric}

\begin{longtable}[]{@{}
  >{\raggedright\arraybackslash}p{(\linewidth - 20\tabcolsep) * \real{0.1573}}
  >{\raggedleft\arraybackslash}p{(\linewidth - 20\tabcolsep) * \real{0.1124}}
  >{\raggedleft\arraybackslash}p{(\linewidth - 20\tabcolsep) * \real{0.1573}}
  >{\raggedleft\arraybackslash}p{(\linewidth - 20\tabcolsep) * \real{0.0787}}
  >{\raggedleft\arraybackslash}p{(\linewidth - 20\tabcolsep) * \real{0.0674}}
  >{\raggedleft\arraybackslash}p{(\linewidth - 20\tabcolsep) * \real{0.0674}}
  >{\raggedleft\arraybackslash}p{(\linewidth - 20\tabcolsep) * \real{0.0787}}
  >{\raggedleft\arraybackslash}p{(\linewidth - 20\tabcolsep) * \real{0.0674}}
  >{\raggedleft\arraybackslash}p{(\linewidth - 20\tabcolsep) * \real{0.0787}}
  >{\raggedleft\arraybackslash}p{(\linewidth - 20\tabcolsep) * \real{0.0674}}
  >{\raggedright\arraybackslash}p{(\linewidth - 20\tabcolsep) * \real{0.0674}}@{}}
\toprule\noalign{}
\begin{minipage}[b]{\linewidth}\raggedright
skim\_variable
\end{minipage} & \begin{minipage}[b]{\linewidth}\raggedleft
n\_missing
\end{minipage} & \begin{minipage}[b]{\linewidth}\raggedleft
complete\_rate
\end{minipage} & \begin{minipage}[b]{\linewidth}\raggedleft
mean
\end{minipage} & \begin{minipage}[b]{\linewidth}\raggedleft
sd
\end{minipage} & \begin{minipage}[b]{\linewidth}\raggedleft
p0
\end{minipage} & \begin{minipage}[b]{\linewidth}\raggedleft
p25
\end{minipage} & \begin{minipage}[b]{\linewidth}\raggedleft
p50
\end{minipage} & \begin{minipage}[b]{\linewidth}\raggedleft
p75
\end{minipage} & \begin{minipage}[b]{\linewidth}\raggedleft
p100
\end{minipage} & \begin{minipage}[b]{\linewidth}\raggedright
hist
\end{minipage} \\
\midrule\noalign{}
\endhead
\bottomrule\noalign{}
\endlastfoot
ID & 0 & 1.0 & 50.50 & 29.01 & 1.0 & 25.75 & 50.5 & 75.25 & 100.0 &
▇▇▇▇▇ \\
Age & 0 & 1.0 & 52.05 & 16.26 & 26.0 & 38.75 & 50.0 & 67.25 & 79.0 &
▇▇▅▇▇ \\
Poids & 10 & 0.9 & 69.38 & 15.59 & 26.8 & 59.93 & 68.9 & 80.65 & 104.1 &
▁▃▇▆▂ \\
Tension & 10 & 0.9 & 132.08 & 18.63 & 100.0 & 120.00 & 130.5 & 143.00 &
185.0 & ▅▇▇▂▁ \\
Cholesterol & 10 & 0.9 & 207.59 & 28.55 & 126.0 & 188.50 & 207.5 &
229.00 & 264.0 & ▁▃▇▇▅ \\
Suivi\_Jours & 0 & 1.0 & 24.83 & 3.34 & 20.0 & 22.00 & 25.0 & 28.00 &
30.0 & ▇▃▃▃▅ \\
Symptom\_Score & 0 & 1.0 & 5.78 & 2.79 & 1.0 & 3.00 & 6.0 & 8.00 & 10.0
& ▆▇▇▇▇ \\
\end{longtable}

\subsection{2.4 Gestion des valeurs
manquantes}\label{gestion-des-valeurs-manquantes}

\begin{Shaded}
\begin{Highlighting}[]
\FunctionTok{colSums}\NormalTok{(}\FunctionTok{is.na}\NormalTok{(data))}
\end{Highlighting}
\end{Shaded}

\begin{verbatim}
##                ID               Age              Sexe             Poids 
##                 0                 0                 0                10 
##           Tension       Cholesterol Groupe_Traitement       Suivi_Jours 
##                10                10                 0                 0 
##     Symptom_Score 
##                 0
\end{verbatim}

\begin{Shaded}
\begin{Highlighting}[]
\NormalTok{data }\OtherTok{\textless{}{-}} \FunctionTok{na.omit}\NormalTok{(data)}
\FunctionTok{nrow}\NormalTok{(data)}
\end{Highlighting}
\end{Shaded}

\begin{verbatim}
## [1] 72
\end{verbatim}

\section{3. Analyse univariée}\label{analyse-univariuxe9e}

\subsection{3.1 Variables quantitatives}\label{variables-quantitatives}

\begin{Shaded}
\begin{Highlighting}[]
\NormalTok{quant\_vars }\OtherTok{\textless{}{-}} \FunctionTok{c}\NormalTok{(}\StringTok{"Age"}\NormalTok{, }\StringTok{"Poids"}\NormalTok{, }\StringTok{"Tension"}\NormalTok{, }\StringTok{"Cholesterol"}\NormalTok{, }\StringTok{"Suivi\_Jours"}\NormalTok{, }\StringTok{"Symptom\_Score"}\NormalTok{)}

\NormalTok{data }\SpecialCharTok{\%\textgreater{}\%}
  \FunctionTok{select}\NormalTok{(}\FunctionTok{all\_of}\NormalTok{(quant\_vars)) }\SpecialCharTok{\%\textgreater{}\%}
  \FunctionTok{summary}\NormalTok{()}
\end{Highlighting}
\end{Shaded}

\begin{verbatim}
##       Age            Poids          Tension       Cholesterol     Suivi_Jours
##  Min.   :26.00   Min.   :26.80   Min.   :100.0   Min.   :126.0   Min.   :20  
##  1st Qu.:38.00   1st Qu.:58.45   1st Qu.:118.5   1st Qu.:192.8   1st Qu.:22  
##  Median :49.50   Median :69.30   Median :128.0   Median :211.0   Median :25  
##  Mean   :51.97   Mean   :68.82   Mean   :130.6   Mean   :209.6   Mean   :25  
##  3rd Qu.:67.25   3rd Qu.:80.35   3rd Qu.:140.0   3rd Qu.:230.2   3rd Qu.:28  
##  Max.   :79.00   Max.   :99.00   Max.   :185.0   Max.   :264.0   Max.   :30  
##  Symptom_Score   
##  Min.   : 1.000  
##  1st Qu.: 4.000  
##  Median : 6.000  
##  Mean   : 5.931  
##  3rd Qu.: 8.000  
##  Max.   :10.000
\end{verbatim}

\begin{Shaded}
\begin{Highlighting}[]
\CommentTok{\# Histogrammes}
\ControlFlowTok{for}\NormalTok{ (v }\ControlFlowTok{in}\NormalTok{ quant\_vars) \{}
  \FunctionTok{print}\NormalTok{(}
    \FunctionTok{ggplot}\NormalTok{(data, }\FunctionTok{aes\_string}\NormalTok{(}\AttributeTok{x =}\NormalTok{ v)) }\SpecialCharTok{+}
      \FunctionTok{geom\_histogram}\NormalTok{(}\AttributeTok{bins =} \DecValTok{20}\NormalTok{, }\AttributeTok{fill =} \StringTok{"steelblue"}\NormalTok{, }\AttributeTok{color =} \StringTok{"white"}\NormalTok{) }\SpecialCharTok{+}
      \FunctionTok{geom\_density}\NormalTok{(}\AttributeTok{alpha =} \FloatTok{0.4}\NormalTok{, }\AttributeTok{color =} \StringTok{"red"}\NormalTok{) }\SpecialCharTok{+}
      \FunctionTok{labs}\NormalTok{(}\AttributeTok{title =} \FunctionTok{paste}\NormalTok{(}\StringTok{"Distribution de"}\NormalTok{, v))}
\NormalTok{  )}
\NormalTok{\}}
\end{Highlighting}
\end{Shaded}

\pandocbounded{\includegraphics[keepaspectratio]{Projet_Statistique_files/figure-latex/univariate-num-1.pdf}}
\pandocbounded{\includegraphics[keepaspectratio]{Projet_Statistique_files/figure-latex/univariate-num-2.pdf}}
\pandocbounded{\includegraphics[keepaspectratio]{Projet_Statistique_files/figure-latex/univariate-num-3.pdf}}
\pandocbounded{\includegraphics[keepaspectratio]{Projet_Statistique_files/figure-latex/univariate-num-4.pdf}}
\pandocbounded{\includegraphics[keepaspectratio]{Projet_Statistique_files/figure-latex/univariate-num-5.pdf}}
\pandocbounded{\includegraphics[keepaspectratio]{Projet_Statistique_files/figure-latex/univariate-num-6.pdf}}

\subsubsection{Test de normalité
(Shapiro-Wilk)}\label{test-de-normalituxe9-shapiro-wilk}

\begin{Shaded}
\begin{Highlighting}[]
\ControlFlowTok{for}\NormalTok{ (v }\ControlFlowTok{in}\NormalTok{ quant\_vars) \{}
  \FunctionTok{cat}\NormalTok{(}\StringTok{"Variable :"}\NormalTok{, v, }\StringTok{"}\SpecialCharTok{\textbackslash{}n}\StringTok{"}\NormalTok{)}
\NormalTok{  x }\OtherTok{\textless{}{-}}\NormalTok{ data[[v]]}
  \ControlFlowTok{if}\NormalTok{ (}\FunctionTok{length}\NormalTok{(x) }\SpecialCharTok{\textgreater{}} \DecValTok{5000}\NormalTok{) x }\OtherTok{\textless{}{-}} \FunctionTok{sample}\NormalTok{(x, }\DecValTok{5000}\NormalTok{)}
  \FunctionTok{print}\NormalTok{(}\FunctionTok{shapiro.test}\NormalTok{(x))}
  \FunctionTok{cat}\NormalTok{(}\StringTok{"}\SpecialCharTok{\textbackslash{}n}\StringTok{{-}{-}{-}{-}{-}{-}{-}{-}{-}{-}{-}{-}{-}{-}{-}{-}{-}{-}{-}{-}{-}{-}{-}{-}{-}{-}{-}{-}{-}{-}{-}{-}{-}}\SpecialCharTok{\textbackslash{}n}\StringTok{"}\NormalTok{)}
\NormalTok{\}}
\end{Highlighting}
\end{Shaded}

\begin{verbatim}
## Variable : Age 
## 
##  Shapiro-Wilk normality test
## 
## data:  x
## W = 0.93446, p-value = 0.001008
## 
## 
## ---------------------------------
## Variable : Poids 
## 
##  Shapiro-Wilk normality test
## 
## data:  x
## W = 0.97775, p-value = 0.2312
## 
## 
## ---------------------------------
## Variable : Tension 
## 
##  Shapiro-Wilk normality test
## 
## data:  x
## W = 0.95439, p-value = 0.01077
## 
## 
## ---------------------------------
## Variable : Cholesterol 
## 
##  Shapiro-Wilk normality test
## 
## data:  x
## W = 0.9828, p-value = 0.4311
## 
## 
## ---------------------------------
## Variable : Suivi_Jours 
## 
##  Shapiro-Wilk normality test
## 
## data:  x
## W = 0.91295, p-value = 0.0001052
## 
## 
## ---------------------------------
## Variable : Symptom_Score 
## 
##  Shapiro-Wilk normality test
## 
## data:  x
## W = 0.94725, p-value = 0.00446
## 
## 
## ---------------------------------
\end{verbatim}

\subsection{3.2 Variables qualitatives}\label{variables-qualitatives}

\begin{Shaded}
\begin{Highlighting}[]
\FunctionTok{table}\NormalTok{(data}\SpecialCharTok{$}\NormalTok{Sexe)}
\end{Highlighting}
\end{Shaded}

\begin{verbatim}
## 
##  F  M 
## 30 42
\end{verbatim}

\begin{Shaded}
\begin{Highlighting}[]
\FunctionTok{table}\NormalTok{(data}\SpecialCharTok{$}\NormalTok{Groupe\_Traitement)}
\end{Highlighting}
\end{Shaded}

\begin{verbatim}
## 
##  A  B 
## 40 32
\end{verbatim}

\begin{Shaded}
\begin{Highlighting}[]
\FunctionTok{ggplot}\NormalTok{(data, }\FunctionTok{aes}\NormalTok{(}\AttributeTok{x =}\NormalTok{ Sexe)) }\SpecialCharTok{+}
  \FunctionTok{geom\_bar}\NormalTok{(}\AttributeTok{fill =} \StringTok{"\#3498db"}\NormalTok{) }\SpecialCharTok{+}
  \FunctionTok{labs}\NormalTok{(}\AttributeTok{title =} \StringTok{"Répartition par sexe"}\NormalTok{, }\AttributeTok{x =} \StringTok{"Sexe"}\NormalTok{, }\AttributeTok{y =} \StringTok{"Effectif"}\NormalTok{)}
\end{Highlighting}
\end{Shaded}

\pandocbounded{\includegraphics[keepaspectratio]{Projet_Statistique_files/figure-latex/univariate-cat-1.pdf}}

\begin{Shaded}
\begin{Highlighting}[]
\FunctionTok{ggplot}\NormalTok{(data, }\FunctionTok{aes}\NormalTok{(}\AttributeTok{x =}\NormalTok{ Groupe\_Traitement)) }\SpecialCharTok{+}
  \FunctionTok{geom\_bar}\NormalTok{(}\AttributeTok{fill =} \StringTok{"\#2ecc71"}\NormalTok{) }\SpecialCharTok{+}
  \FunctionTok{labs}\NormalTok{(}\AttributeTok{title =} \StringTok{"Répartition par groupe de traitement"}\NormalTok{, }\AttributeTok{x =} \StringTok{"Groupe"}\NormalTok{, }\AttributeTok{y =} \StringTok{"Effectif"}\NormalTok{)}
\end{Highlighting}
\end{Shaded}

\pandocbounded{\includegraphics[keepaspectratio]{Projet_Statistique_files/figure-latex/univariate-cat-2.pdf}}

\section{4. Analyse bivariée}\label{analyse-bivariuxe9e}

\subsection{4.1 Corrélations entre variables
quantitatives}\label{corruxe9lations-entre-variables-quantitatives}

\begin{Shaded}
\begin{Highlighting}[]
\NormalTok{cor\_matrix }\OtherTok{\textless{}{-}} \FunctionTok{cor}\NormalTok{(data[, quant\_vars])}
\NormalTok{cor\_matrix}
\end{Highlighting}
\end{Shaded}

\begin{verbatim}
##                        Age       Poids      Tension Cholesterol Suivi_Jours
## Age            1.000000000  0.07581081 -0.007322607 -0.06885022 -0.06815034
## Poids          0.075810814  1.00000000 -0.110141073  0.06623760  0.05151972
## Tension       -0.007322607 -0.11014107  1.000000000  0.03471370 -0.03240724
## Cholesterol   -0.068850218  0.06623760  0.034713702  1.00000000  0.02892325
## Suivi_Jours   -0.068150342  0.05151972 -0.032407243  0.02892325  1.00000000
## Symptom_Score  0.168343356  0.09557221 -0.119349464 -0.03796789 -0.11738130
##               Symptom_Score
## Age              0.16834336
## Poids            0.09557221
## Tension         -0.11934946
## Cholesterol     -0.03796789
## Suivi_Jours     -0.11738130
## Symptom_Score    1.00000000
\end{verbatim}

\begin{Shaded}
\begin{Highlighting}[]
\NormalTok{GGally}\SpecialCharTok{::}\FunctionTok{ggcorr}\NormalTok{(data[, quant\_vars], }\AttributeTok{label =} \ConstantTok{TRUE}\NormalTok{)}
\end{Highlighting}
\end{Shaded}

\pandocbounded{\includegraphics[keepaspectratio]{Projet_Statistique_files/figure-latex/correlation-1.pdf}}

\subsubsection{Exemple : Corrélation entre âge et score de
symptômes}\label{exemple-corruxe9lation-entre-uxe2ge-et-score-de-symptuxf4mes}

\begin{Shaded}
\begin{Highlighting}[]
\FunctionTok{cor.test}\NormalTok{(data}\SpecialCharTok{$}\NormalTok{Age, data}\SpecialCharTok{$}\NormalTok{Symptom\_Score)}
\end{Highlighting}
\end{Shaded}

\begin{verbatim}
## 
##  Pearson's product-moment correlation
## 
## data:  data$Age and data$Symptom_Score
## t = 1.4289, df = 70, p-value = 0.1575
## alternative hypothesis: true correlation is not equal to 0
## 95 percent confidence interval:
##  -0.0658951  0.3849971
## sample estimates:
##       cor 
## 0.1683434
\end{verbatim}

\begin{Shaded}
\begin{Highlighting}[]
\FunctionTok{ggplot}\NormalTok{(data, }\FunctionTok{aes}\NormalTok{(}\AttributeTok{x =}\NormalTok{ Age, }\AttributeTok{y =}\NormalTok{ Symptom\_Score)) }\SpecialCharTok{+}
  \FunctionTok{geom\_point}\NormalTok{() }\SpecialCharTok{+}
  \FunctionTok{geom\_smooth}\NormalTok{(}\AttributeTok{method =} \StringTok{"lm"}\NormalTok{, }\AttributeTok{color =} \StringTok{"red"}\NormalTok{) }\SpecialCharTok{+}
  \FunctionTok{labs}\NormalTok{(}\AttributeTok{title =} \StringTok{"Âge vs Score de symptômes"}\NormalTok{)}
\end{Highlighting}
\end{Shaded}

\pandocbounded{\includegraphics[keepaspectratio]{Projet_Statistique_files/figure-latex/cor-age-1.pdf}}

\subsection{4.2 Comparaison de moyennes (score selon le
traitement)}\label{comparaison-de-moyennes-score-selon-le-traitement}

\begin{Shaded}
\begin{Highlighting}[]
\FunctionTok{ggplot}\NormalTok{(data, }\FunctionTok{aes}\NormalTok{(}\AttributeTok{x =}\NormalTok{ Groupe\_Traitement, }\AttributeTok{y =}\NormalTok{ Symptom\_Score, }\AttributeTok{fill =}\NormalTok{ Groupe\_Traitement)) }\SpecialCharTok{+}
  \FunctionTok{geom\_boxplot}\NormalTok{() }\SpecialCharTok{+}
  \FunctionTok{labs}\NormalTok{(}\AttributeTok{title =} \StringTok{"Score selon le groupe de traitement"}\NormalTok{)}
\end{Highlighting}
\end{Shaded}

\pandocbounded{\includegraphics[keepaspectratio]{Projet_Statistique_files/figure-latex/anova-1.pdf}}

\begin{Shaded}
\begin{Highlighting}[]
\CommentTok{\# Test ANOVA}
\NormalTok{anova\_test }\OtherTok{\textless{}{-}} \FunctionTok{aov}\NormalTok{(Symptom\_Score }\SpecialCharTok{\textasciitilde{}}\NormalTok{ Groupe\_Traitement, }\AttributeTok{data =}\NormalTok{ data)}
\FunctionTok{summary}\NormalTok{(anova\_test)}
\end{Highlighting}
\end{Shaded}

\begin{verbatim}
##                   Df Sum Sq Mean Sq F value Pr(>F)
## Groupe_Traitement  1    2.6   2.584   0.366  0.547
## Residuals         70  494.1   7.058
\end{verbatim}

\begin{Shaded}
\begin{Highlighting}[]
\CommentTok{\# Test non paramétrique}
\FunctionTok{kruskal.test}\NormalTok{(Symptom\_Score }\SpecialCharTok{\textasciitilde{}}\NormalTok{ Groupe\_Traitement, }\AttributeTok{data =}\NormalTok{ data)}
\end{Highlighting}
\end{Shaded}

\begin{verbatim}
## 
##  Kruskal-Wallis rank sum test
## 
## data:  Symptom_Score by Groupe_Traitement
## Kruskal-Wallis chi-squared = 0.28129, df = 1, p-value = 0.5959
\end{verbatim}

\subsection{4.3 Variables qualitatives (Sexe vs Groupe de
traitement)}\label{variables-qualitatives-sexe-vs-groupe-de-traitement}

\begin{Shaded}
\begin{Highlighting}[]
\NormalTok{tab }\OtherTok{\textless{}{-}} \FunctionTok{table}\NormalTok{(data}\SpecialCharTok{$}\NormalTok{Sexe, data}\SpecialCharTok{$}\NormalTok{Groupe\_Traitement)}
\NormalTok{tab}
\end{Highlighting}
\end{Shaded}

\begin{verbatim}
##    
##      A  B
##   F 14 16
##   M 26 16
\end{verbatim}

\begin{Shaded}
\begin{Highlighting}[]
\FunctionTok{chisq.test}\NormalTok{(tab)}
\end{Highlighting}
\end{Shaded}

\begin{verbatim}
## 
##  Pearson's Chi-squared test with Yates' continuity correction
## 
## data:  tab
## X-squared = 1.0864, df = 1, p-value = 0.2973
\end{verbatim}

\section{5. Analyse multivariée}\label{analyse-multivariuxe9e}

\subsection{5.1 Régression linéaire
multiple}\label{ruxe9gression-linuxe9aire-multiple}

\begin{Shaded}
\begin{Highlighting}[]
\NormalTok{model }\OtherTok{\textless{}{-}} \FunctionTok{lm}\NormalTok{(Symptom\_Score }\SpecialCharTok{\textasciitilde{}}\NormalTok{ Age }\SpecialCharTok{+}\NormalTok{ Sexe }\SpecialCharTok{+}\NormalTok{ Poids }\SpecialCharTok{+}\NormalTok{ Tension }\SpecialCharTok{+}
\NormalTok{              Cholesterol }\SpecialCharTok{+}\NormalTok{ Groupe\_Traitement }\SpecialCharTok{+}\NormalTok{ Suivi\_Jours, }\AttributeTok{data =}\NormalTok{ data)}

\FunctionTok{summary}\NormalTok{(model)}
\end{Highlighting}
\end{Shaded}

\begin{verbatim}
## 
## Call:
## lm(formula = Symptom_Score ~ Age + Sexe + Poids + Tension + Cholesterol + 
##     Groupe_Traitement + Suivi_Jours, data = data)
## 
## Residuals:
##     Min      1Q  Median      3Q     Max 
## -5.4514 -2.0247  0.2219  2.0510  4.5342 
## 
## Coefficients:
##                     Estimate Std. Error t value Pr(>|t|)  
## (Intercept)         9.207592   4.656931   1.977   0.0523 .
## Age                 0.022730   0.019950   1.139   0.2588  
## SexeM              -0.319888   0.727147  -0.440   0.6615  
## Poids               0.009679   0.022673   0.427   0.6709  
## Tension            -0.017937   0.018220  -0.984   0.3286  
## Cholesterol        -0.001958   0.011546  -0.170   0.8659  
## Groupe_TraitementB -0.352014   0.668319  -0.527   0.6002  
## Suivi_Jours        -0.081172   0.095565  -0.849   0.3988  
## ---
## Signif. codes:  0 '***' 0.001 '**' 0.01 '*' 0.05 '.' 0.1 ' ' 1
## 
## Residual standard error: 2.691 on 64 degrees of freedom
## Multiple R-squared:  0.06689,    Adjusted R-squared:  -0.03517 
## F-statistic: 0.6554 on 7 and 64 DF,  p-value: 0.7085
\end{verbatim}

\subsubsection{Interprétation
simplifiée}\label{interpruxe9tation-simplifiuxe9e}

\begin{Shaded}
\begin{Highlighting}[]
\NormalTok{broom}\SpecialCharTok{::}\FunctionTok{tidy}\NormalTok{(model)}
\end{Highlighting}
\end{Shaded}

\begin{verbatim}
## # A tibble: 8 x 5
##   term               estimate std.error statistic p.value
##   <chr>                 <dbl>     <dbl>     <dbl>   <dbl>
## 1 (Intercept)         9.21       4.66       1.98   0.0523
## 2 Age                 0.0227     0.0200     1.14   0.259 
## 3 SexeM              -0.320      0.727     -0.440  0.661 
## 4 Poids               0.00968    0.0227     0.427  0.671 
## 5 Tension            -0.0179     0.0182    -0.984  0.329 
## 6 Cholesterol        -0.00196    0.0115    -0.170  0.866 
## 7 Groupe_TraitementB -0.352      0.668     -0.527  0.600 
## 8 Suivi_Jours        -0.0812     0.0956    -0.849  0.399
\end{verbatim}

\subsection{5.2 Vérification des hypothèses du
modèle}\label{vuxe9rification-des-hypothuxe8ses-du-moduxe8le}

\begin{Shaded}
\begin{Highlighting}[]
\FunctionTok{par}\NormalTok{(}\AttributeTok{mfrow =} \FunctionTok{c}\NormalTok{(}\DecValTok{2}\NormalTok{, }\DecValTok{2}\NormalTok{))}
\FunctionTok{plot}\NormalTok{(model)}
\end{Highlighting}
\end{Shaded}

\pandocbounded{\includegraphics[keepaspectratio]{Projet_Statistique_files/figure-latex/diagnostic-1.pdf}}

\begin{Shaded}
\begin{Highlighting}[]
\FunctionTok{par}\NormalTok{(}\AttributeTok{mfrow =} \FunctionTok{c}\NormalTok{(}\DecValTok{1}\NormalTok{, }\DecValTok{1}\NormalTok{))}
\end{Highlighting}
\end{Shaded}

\section{6. Limites et perspectives}\label{limites-et-perspectives}

\begin{itemize}
\tightlist
\item
  Taille d'échantillon limitée
\item
  Variables non observées (régime, antécédents\ldots)
\item
  Données manquantes ou non normalisées
\item
  Perspectives : ajout de variables, modèles non linéaires, tests
  robustes
\end{itemize}

\section{7. Conclusion}\label{conclusion}

Cette étude montre les relations entre les caractéristiques des patients
et leur état de santé. Les analyses descriptives, corrélationnelles et
de régression permettent de formuler et vérifier des hypothèses
statistiques fiables.

\end{document}
